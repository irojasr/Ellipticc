\documentclass[12pt]{memoir}

\def\nsemestre {V}
\def\nterm {Verano}
\def\nyear {2021}
\def\nprofesor {Adri\'an Barquero y Dar\'io Mena}
\def\nsigla {MA0609}
\def\nsiglahead {T\'opicos en Teor\'ia de N\'umeros}
\def\darktheme {}

\makeatletter
\ifx \nauthor\undefined
  \def\nauthor{Ignacio Rojas}
\else
\fi

\ifx \nextra \undefined
\author{Basado en las clases impartidas por \nprofesor \\\small Notas tomadas por \nauthor}
\else
\author{\nauthor}
\fi
\date{\nterm\ \nyear}

%%%%%%%%%%%%%
%% 1. Pacotes
%%%%%%%%%%%%%

\usepackage{alltt}
\usepackage{amsfonts}
\usepackage{amsmath}
\usepackage{amssymb}
\usepackage{amsthm}
\usepackage{algorithm}
\usepackage[noend]{algpseudocode}
\usepackage{array}
\newcommand\hmmax{0} % default 3
\newcommand\bmmax{0} % default 4 %%tex.se/3676,219310
%\usepackage{bbold}
\usepackage{bm}
\usepackage{booktabs}
%\usepackage{caption}
\usepackage{cancel}
%\usepackage{dsfont}
\usepackage{esint}
\usepackage{fancyhdr}
\usepackage{graphicx}
\usepackage[utf8]{inputenc}
\usepackage{listings}
\usepackage{mathabx}
\usepackage[cal=euler]{mathalfa}
%\usepackage[cal=euler,frak=euler]{mathalfa} % mathcal (JIRR) precisabamos correr initexmf --mkmaps en cmd JCVDG
\usepackage{mathdots}
\usepackage{mathrsfs}
%\usepackage{mathtools}
\usepackage{microtype}
\usepackage{multicol}
\usepackage{multirow}
\usepackage[theoremfont,largesc,tighter,osf]{newpxtext} %JCV Diff
\let\widering\undefined
%\usepackage[bigdelims,vvarbb]{newpxmath} %JCVDG
%por alguna razón esto afectaba las tildes en \min, \lim y demás
%\usepackage{pdflscape}
\usepackage{pgfplots}
\usepackage{physics}
\usepackage{siunitx}
\usepackage{slashed}
\usepackage{stmaryrd}
%\usepackage{subfigure}
\usepackage{tabularx}
\usepackage[breakable,skins]{tcolorbox}
\usepackage{textcomp} %%JCVDG
\usepackage{tikz}
\usepackage{tkz-euclide}
\usepackage[normalem]{ulem}
\usepackage[all]{xy}
\usepackage{imakeidx}
\usepackage[spanish]{babel}

%%%%%%%%%%%%%%%%%%%%
%% 2. Document Setup
%%%%%%%%%%%%%%%%%%%%

\definecolor{MATLABgreen}{RGB}{28,172,0} % color values Red, Green, Blue
\definecolor{MATLABlila}{RGB}{170,55,241}
\definecolor{dankBlue}{RGB}{51,60,77} % color values Red, Green, Blue
\definecolor{dankBlueLite}{RGB}{82,97,125} % color values Red, Green, Blue

\ifx \nextra \undefined
\makeindex[intoc, title=Índice Analítico] %Título de índice analítico
%El índice general es aquel en el que se indican los capítulos, títulos y subtítulos del libro.
%Índice onomástico es donde aparece el nombre de personas mencionadas en el texto, por orden alfabético con el número de las páginas donde aparecen.
%El índice analítico se refiere a los temas y conceptos que aparecen en el libro
\indexsetup{othercode={\fancyhead[LE]{\emph{Índice Analítico}}}}
  \usepackage[pdftex,
    hidelinks,
    pdfauthor={\nauthor},
    pdfsubject={Notas Escuela de Matem\'aticas UCR: \nsiglahead\ \nsemestre-\nyear},
    pdftitle={Semestre \nsemestre\ - \nsigla},
  pdfkeywords={UCR Costa Rica Matem\'aticas Mate \nsemestre\ \nterm\ \nyear\ \nsigla}]{hyperref}
  \title{\nsigla\ --- \nsiglahead}
\else
  \usepackage[pdftex,
     hidelinks,
    pdfauthor={\nauthor},
    pdfsubject={Notas Escuela de Matem\'aticas UCR: \nsiglahead\ \nsemestre-\nyear},
    pdftitle={Semestre \nsemestre\ - \nsigla},
  pdfkeywords={UCR Costa Rica Matem\'aticas Mate \nsemestre\ \nterm\ \nyear\ \nsigla\ \nextra}]{hyperref}

  \title{\nsigla\ --- \nsiglahead \\ {\Large \nextra}}
  \renewcommand\printindex{}
\fi

\pgfplotsset{compat=1.12}


\pagestyle{fancy}
\setlength{\headheight}{15.72pt} %preceding warning said make it at least this

\ifx \nsiglahead \undefined
\def\nsiglahead{\nsigla}
\fi

\lhead{} %%%empty lhead
\rfoot{\thepage}

\ifx \nextra \undefined
  \chead{
    \ifnum\thepage=1
    \else
      \textbf{Notas \nsiglahead\ \nsemestre-\nyear}
    \fi}
  \rhead{}%\firstxmark} % Top right header
\else
%    \chead{
%    \ifnum\thepage=1
%    \else
%      \textbf{Notas \nsiglahead\ \nsemestre-\nyear \ (\nextra)}
%    \fi}
     \chead{
       \textbf{\nextra\ \nsigla\ \nsemestre-\nyear}
     }
     \rhead{
       \textbf{\nauthor}
     }
\fi
\lfoot{}%\lastxmark} % Bottom left footer
\cfoot{} % Bottom center footer

\usetikzlibrary{arrows.meta}
\usetikzlibrary{decorations.markings}
\usetikzlibrary{decorations.pathmorphing}
\usetikzlibrary{positioning}
\usetikzlibrary{fadings}
\usetikzlibrary{intersections}
\usetikzlibrary{cd}

\ifx \nhtml \undefined
\else
  \renewcommand\printindex{}
  \DisableLigatures[f]{family = *}
  \let\Contentsline\contentsline
  \renewcommand\contentsline[3]{\Contentsline{#1}{#2}{}}
  \renewcommand{\@dotsep}{10000}
  \newlength\currentparindent
  \setlength\currentparindent\parindent

  \newcommand\@minipagerestore{\setlength{\parindent}{\currentparindent}}
  \usepackage[active,tightpage,pdftex]{preview}
  \renewcommand{\PreviewBorder}{0.1cm}

  \newenvironment{stretchpage}%
  {\begin{preview}\begin{minipage}{\hsize}}%
    {\end{minipage}\end{preview}}
  \AtBeginDocument{\begin{stretchpage}}
  \AtEndDocument{\end{stretchpage}}

  \newcommand{\@@newpage}{\end{stretchpage}\begin{stretchpage}}

  \let\@real@section\section
  \renewcommand{\section}{\@@newpage\@real@section}
  \let\@real@subsection\subsection
  \renewcommand{\subsection}{\@ifstar{\@real@subsection*}{\@@newpage\@real@subsection}}
\fi
\ifx \ntrim \undefined
\usepackage[shortlabels]{enumitem} %mfw package order matters por savetrees
\else
  \usepackage{geometry}
  \geometry{
    papersize={379pt, 699pt},
    textwidth=345pt,
    textheight=596pt,
    left=17pt,
    top=54pt,
    right=17pt
  }
 \usepackage[extreme]{savetrees}
\fi

\ifx \darktheme\undefined
\else
\pagecolor[rgb]{0.2,0.231,0.302}%{0.23,0.258,0.321}
\color[rgb]{1,1,1}
\fi

\ifx \nextra \undefined
\let\@real@maketitle\maketitle
\renewcommand{\maketitle}{\@real@maketitle\begin{center}\begin{minipage}[c]{0.9\textwidth}\centering\footnotesize Estas notas no están respaldadas por los profesores y han sido modificadas (a menudo de manera significativa) después de las clases. No están lejos de ser representaciones precisas de lo que realmente se dio en clase y en particular todos los errores son casi seguramente míos.\end{minipage}\end{center}}
\else
\fi

\def\moverlay{\mathpalette\mov@rlay}
\def\mov@rlay#1#2{\leavevmode\vtop{%
   \baselineskip\z@skip \lineskiplimit-\maxdimen
   \ialign{\hfil$\m@th#1##$\hfil\cr#2\crcr}}}
\newcommand{\charfusion}[3][\mathord]{
    #1{\ifx#1\mathop\vphantom{#2}\fi
        \mathpalette\mov@rlay{#2\cr#3}
      }
    \ifx#1\mathop\expandafter\displaylimits\fi}

%%%%%%%%%%%%%%%%%%%%%%%%%%%%%%
%% 2.1 Some internal machinery
%%%%%%%%%%%%%%%%%%%%%%%%%%%%%%

\makeatletter
\renewcommand{\section}{\@startsection{section}{1}{\z@}%
							 {-3.25ex \@plus -1ex \@minus -.2ex}%
							 {1.5ex \@plus.2ex}%
							 {\normalfont\large\bfseries}}
\renewcommand{\subsection}{\@startsection{subsection}{2}{\z@}%
							 {-3.25ex \@plus -1ex \@minus -.2ex}%
							 {1.5ex \@plus .2ex}%
               {\normalfont\normalsize\bfseries}}
\newcommand*{\defeq}{\mathrel{\rlap{%
             \raisebox{0.3ex}{$\m@th\cdot$}}%
             \raisebox{-0.3ex}{$\m@th\cdot$}}%
                    =}
\makeatother
\ifx\ntrim\undefined
\newcommand{\coursetitle}{\nsigla: \nsiglahead}
\ifx\nextra\undefined
\pagestyle{ruled}
\makeoddhead{ruled}{\coursetitle}{}{\rightmark}
\else\fi
\settypeblocksize{49pc}{37pc}{*}
\setlrmargins{*}{*}{1.2}
\setulmargins{*}{*}{0.8}
\setheadfoot{16pt}{30pt}
\setheaderspaces{*}{1.5pc}{1}
\setmarginnotes{1pt}{1pt}{1pt}
\checkandfixthelayout

\setlength{\unitlength}{3pt}
\setlength{\hfuzz}{1pt}

\setlength{\fboxsep}{6pt}

\setlength{\footskip}{17pt}

\linespread{1.1}
\else\fi
\renewcommand{\cftdotsep}{\cftnodots} %%% no dots in ToC
\setpnumwidth{2em}  %%% width of page-number box in ToC


\newcommand{\stophere}{\relax} %% can be changed to `\endinput'
% \newcommand{\stophere}{\endinput} %% can be changed to `\relax'


\DeclareRobustCommand{\qned}{\ifmmode
  \else \leavevmode\unskip\penalty9999 \hbox{}\nobreak\hfill \fi
  \quad\hbox{\qnedsymbol}}
\newcommand{\qnedsymbol}{$\boxminus$} %% No-proofs end with `\qned'

\DeclareRobustCommand{\qef}{\ifmmode
  \else \leavevmode\unskip\penalty9999 \hbox{}\nobreak\hfill \fi
  \quad\hbox{\qefsymbol}}
\newcommand{\qefsymbol}{$\lozenge$} %% Examples end with `\qef'
\def\enddefn{\qef\endtrivlist}      %% `\qef' automático en defns
\def\endejem{\qef\endtrivlist}      %% `\qef' automático en ejemplos

\newcommand{\hideqed}{\renewcommand{\qed}{}} %% to suppress `\qed'
\newcommand{\hideqef}{\renewcommand{\qef}{}} %% to suppress `\qef'

% \newcommand{\ldbrack}{\ensuremath{[\mskip-2.5mu[}} %% corchetes [[
% \newcommand{\rdbrack}{\ensuremath{]\mskip-2.5mu]}} %% corchetes ]]

\newcommand{\stroke}{\mathbin|}     %% (for `\bbraket' and such)

\newcommand{\rtri}{\blacktriangleright} %% (for `\marker' and such)
\newcommand{\tribar}{|\mkern-2mu|\mkern-2mu|} %% norma triple: |||


%% Formatting changes:

\renewcommand{\labelitemi}{$\diamond$} %% instead of bullets

\renewcommand{\theenumi}{\alph{enumi}}  %% use lowercase letters
\renewcommand{\labelenumi}{\textup{(\theenumi)}} %% inside parentheses


%%%%%%%%%%%%%%%%%%%%%%%%%%%
%% 3. Theorems and suchlike
%%%%%%%%%%%%%%%%%%%%%%%%%%%

% Theorems
\theoremstyle{definition}
\newtheorem*{aim}{Aim}
\newtheorem*{axiom}{Axiom}
\newtheorem*{claim}{Claim}
\newtheorem*{conjecture}{Conjecture}
\newtheorem*{law}{Law}
\newtheorem*{question}{Question}
\newtheorem*{rrule}{Rule}
\newtheorem*{assumption}{Assumption}

% \renewcommand{\labelitemi}{--}
% \renewcommand{\labelitemii}{$\circ$}
% \renewcommand{\labelenumi}{(\roman{*})}

%\let\stdsection\section
%\renewcommand\section{\newpage\stdsection}

\newcommand\qedsym{\hfill\ensuremath{\square}}
% Strike through
\def\st{\bgroup \ULdepth=-.55ex \ULset}

%%%%%%%%% === My T Color Box === %%%%%%%%%%%%%%

\ifx \darktheme\undefined
\newtcolorbox{ptcb}{
colframe = black,
colback = white,
breakable,
enhanced
}
\newtcolorbox{ptcbp}{
colframe = black,
colback = white,
coltitle = black,
colbacktitle = black!40,
title = Prueba,
breakable,
enhanced
}
\newtcolorbox{ptcbr}{
colframe = blue,
colback = white,
coltitle = blue,
colbacktitle = blue!40,
title = Respuesta,
breakable,
enhanced
}
\else
\newtcolorbox{ptcb}{
colframe = white,
colback = dankBlue,
colupper = white,
breakable,
enhanced
}
\newtcolorbox{ptcbp}{
colframe = white,
colback = dankBlue,
colupper = white,
coltitle = white,
colbacktitle = dankBlueLite,
title = Prueba,
breakable,
enhanced
}
\newtcolorbox{ptcbr}{
colframe = white,
colback = white,
coltitle = blue,
colbacktitle = blue!40,
title = Respuesta,
breakable,
enhanced
}
\fi







%%%%%%%%% === Listings === %%%%%%%%%%%%%%
\lstset{basicstyle=\ttfamily,breaklines=true}

\lstset{language=Matlab,%
    %basicstyle=\color{red},
    breaklines=true,%
    morekeywords={matlab2tikz},
    keywordstyle=\color{blue},%
    morekeywords=[2]{1}, keywordstyle=[2]{\color{black}},
    identifierstyle=\color{black},%
    stringstyle=\color{MATLABlila},
    commentstyle=\color{MATLABgreen},%
    showstringspaces=false,%without this there will be a symbol in the places where there is a space
    numbers=left,%
    numberstyle={\tiny \color{black}},% size of the numbers
    numbersep=9pt, % this defines how far the numbers are from the text
   % emph=[1]{for,end,break,function,if,elseif,else},emphstyle=[1]\color{blue}, %some words to emphasise
    %emph=[2]{word1,word2}, emphstyle=[2]{style},
}

%%%%%%%%% === Theorems and suchlike === %%%%%%%%%%%%%%

\theoremstyle{plain}
\ifx \nextra \undefined
\newtheorem{Th}{Teorema}[section]      %%% Theorem 1.1.1
\newtheorem{Tmon}[Th]{Teoremón}
\newtheorem{Prop}[Th]{Proposición}     %%% Proposition 1.1.2
\newtheorem{Lem}[Th]{Lema}             %%% Lemma 1.1.3
\newtheorem{Cor}[Th]{Corolario}        %%% Corollary 1.1.4
\else
\newtheorem{Th}{Teorema}               %%% Theorem 1.1.1
\newtheorem{Tmon}{Teoremón}
\newtheorem{Prop}{Proposición}         %%% Proposition 1.1.2
\newtheorem{Lem}{Lema}                 %%% Lemma 3
\newtheorem{Cor}{Corolario}            %%% Corollary 4
\fi
\newtheorem*{nonum-Th}{Teorema}        %%% No-numbered Theorem
\newtheorem*{nonum-Cor}{Corolario}     %%% No-numbered Corollary

\theoremstyle{definition}
\ifx \nextra \undefined
\newtheorem{Def}[Th]{Definición}       %%% Definition 1.1.5
\newtheorem{Ex}[Th]{Ejemplo}           %%% Example 1.1.6
\newtheorem{Ej}[Th]{Ejercicio}         %%% Ejercicio 1.1.7
\else
\newtheorem{Def}{Definición}           %%% Definition 5
\newtheorem{Ex}{Ejemplo}               %%% Example 6
\newtheorem{Ej}{Ejercicio}             %%% Ejercicio 7
\fi
\newtheorem{Hec}[Th]{Hecho}            %%% Hecho 1.1.8
\newtheorem*{nonum-Def}{Definición}    %%% No number Definition
\newtheorem*{nonum-Ex}{Ejemplo}        %%% No number Example
\newtheorem*{nonum-Ej}{Ejercicio}      %%% No number Ejercicio
\newtheorem*{nonum-Hec}{Hecho}         %%% No number Fact

\theoremstyle{remark}
\newtheorem{Rmk}[Th]{Observación}      %%%Remark 1.1.9
\newtheorem*{nonum-Rmk}{Observación}         %%% No number Fact
\newtheorem*{Notn}{Notaci\'on}        %% Notaciones
\newtheorem*{Warn}{Advertencia}       %% Advertencias

\numberwithin{equation}{section}

\setlength{\parindent}{3ex}


%%%%%%%%%%%%%%%%%%%%%%%%%
%%%%% Maths Symbols %%%%%
%%%%%%%%%%%%%%%%%%%%%%%%%

% not-math
\newcommand{\bolds}[1]{{\bfseries #1}}
\newcommand{\cat}[1]{\mathsf{#1}}
\newcommand{\ph}{\,\cdot\,}
\newcommand{\term}[1]{\un{#1}\index{#1}}
\newcommand{\phantomeq}{\hphantom{{}={}}}
\newcommand{\ttt}{\texttt}
\newcommand{\red}[1]{\textcolor{red}{#1}}
\newcommand{\prp}[1]{\textcolor{purple}{#1}}
\newcommand{\blu}[1]{\textcolor{blue}{#1}}

%functions
\DeclareMathOperator{\sgn}{sgn}
\newcommand*{\Cdot}{{\raisebox{-0.25ex}{\scalebox{1.5}{$\cdot$}}}}      %% cdot más grande
\newcommand{\ind}{\mathbf{1}}       %%%indicator function
\newcommand{\mm}{\mathfrak{m}}      %%%metric


% Greek letters:

\newcommand{\al}{\alpha}                %% short for  \alpha
\newcommand{\bt}{\beta}                 %% short for  \beta
\newcommand{\Dl}{\Delta}                %% short for  \Delta
\newcommand{\dl}{\delta}                %% short for  \delta
\newcommand{\eps}{\varepsilon}          %% short for  \varepsilon
\newcommand{\Ga}{\Gamma}                %% short for  \Gamma
\newcommand{\ga}{\gamma}                %% short for  \gamma
\newcommand{\kp}{\kappa}                %% short for  \kappa
\newcommand{\La}{\Lambda}               %% short for  \Lambda
\newcommand{\la}{\lambda}               %% short for  \lambda
\newcommand{\Om}{\Omega}                %% short for  \Omega
\newcommand{\om}{\omega}                %% short for  \omega
\newcommand{\Sg}{\Sigma}                %% short for  \Sigma
\newcommand{\sg}{\sigma}                %% short for  \sigma
\newcommand{\Te}{\Theta}                %% short for  \Theta
\newcommand{\te}{\theta}                %% short for  \theta
\newcommand{\ups}{\upsilon}             %% short for  \upsilon
\newcommand{\vf}{\varphi}               %% short for  \varphi
\newcommand{\ze}{\zeta}                 %% short for  \zeta
\newcommand{\vsg}{\varsigma}            %% short for  \varsigma
\newcommand{\vte}{\vartheta}            %% short for  \vartheta

%Boldface letters

\newcommand{\bA}{\mathbb{A}}        %% antisimetrizador
\newcommand{\bB}{\mathbb{B}}        %% bola unitaria
\newcommand{\bC}{\mathbb{C}}    %%% números complejos
\newcommand{\bCP}{\mathbb{CP}}  %%% espacio proyectivo complejo
\newcommand{\bD}{\mathbb{D}}        %% Poincaré disk
\newcommand{\bE}{\mathbb{E}}
\newcommand{\bF}{\mathbb{F}}        %% un cuerpo
\newcommand{\bH}{\mathbb{H}}        %% cuaterniones
\newcommand{\bK}{\mathbb{K}}            %% ein korper
\newcommand{\bN}{\mathbb{N}}    %%% números naturales
\newcommand{\bP}{\mathbb{P}}        %% números enteros positivos
\newcommand{\bQ}{\mathbb{Q}}    %%% números racionales
\newcommand{\bR}{\mathbb{R}}    %%% números reales
\newcommand{\bRP}{\mathbb{RP}}  %%% espacio proyectivo real
\newcommand{\bS}{\mathbb{S}}    %%% esfera
\newcommand{\bT}{\mathbb{T}}        %% círculo o toro
\newcommand{\bZ}{\mathbb{Z}}    %%% números enteros

%Script letters:

\newcommand{\cA}{\mathcal{A}}           %% formas diferenciales
\newcommand{\cB}{\mathcal{B}}           %% una base vectorial
\newcommand{\cC}{\mathcal{C}}           %% otra base vectorial
\newcommand{\cD}{\mathcal{D}}           %% funciones de prueba
\newcommand{\cE}{\mathcal{E}}           %% un modulo proyectivo
\newcommand{\cF}{\mathcal{F}}           %% espacio de Fock
\newcommand{\cG}{\mathcal{G}}           %% funtor de Gelfand
\newcommand{\cH}{\mathcal{H}}           %% espacio de Hilbert
\newcommand{\cI}{\mathcal{I}}           %% un funtor de inclusion
\newcommand{\cJ}{\mathcal{J}}           %% otro funtor
\newcommand{\cK}{\mathcal{K}}           %% otro espacio de Hilbert
\newcommand{\cL}{\mathcal{L}}           %% operadores lineales
\newcommand{\cM}{\mathcal{M}}           %% multiplicadores
\newcommand{\cN}{\mathcal{N}}           %% funciones nulas
\newcommand{\cO}{\mathcal{O}}           %% funciones de crec-to lento
\newcommand{\cP}{\mathcal{P}}           %% una particion
\newcommand{\cR}{\mathcal{R}}           %% funciones representativas
\newcommand{\cQ}{\mathcal{Q}}           %% otra particion
\newcommand{\cS}{\mathcal{S}}           %% funciones de Schwartz
\newcommand{\cT}{\mathcal{T}}           %% una topologia
\newcommand{\cU}{\mathcal{U}}           %% cubrimiento abierto
\newcommand{\cV}{\mathcal{V}}           %% vecindarioas
\newcommand{\cW}{\mathcal{W}}           %% grupo de Weyl

%%% Fraktur letters:

\newcommand{\gA}{\mathfrak{A}}      %% un atlas
\newcommand{\g}{\mathfrak{g}}       %% un álgebra de Lie
\newcommand{\gB}{\mathfrak{B}}      %% otro atlas
\newcommand{\ggl}{\mathfrak{gl}}    %% álg de Lie general lineal
\newcommand{\gsl}{\mathfrak{sl}}    %% álg de Lie especial lineal
\newcommand{\gso}{\mathfrak{so}}    %% álg de Lie especial ortogonal
\newcommand{\gsu}{\mathfrak{su}}    %% álg de Lie especial unitaria
\newcommand{\gX}{\mathfrak{X}}      %% campos vectoriales

%%% Roman letters:

\newcommand{\dR}{\mathrm{dR}}       %% cohomología de de Rham
\newcommand{\rGL}{\mathrm{GL}}      %% grupo general lineal
\newcommand{\rO}{\mathrm{O}}        %% grupo ortogonal
\newcommand{\rSL}{\mathrm{SL}}      %% grupo especial lineal
\newcommand{\rSO}{\mathrm{SO}}      %% grupo ortogonal especial
\newcommand{\rSp}{\mathrm{Sp}}      %% grupo simpléctico
\newcommand{\rSU}{\mathrm{SU}}      %% grupo unitario especial
\newcommand{\rU}{\mathrm{U}}        %% grupo unitario
\newcommand{\rUH}{\mathrm{UH}}      %% cuaterniones unitarias
\newcommand{\rT}{\mathrm{T}}        %% grupo triangular

% Sanserif letters:

\newcommand{\sA}{\mathsf{A}}            %% algebras de Lie A_n
\newcommand{\sB}{\mathsf{B}}            %% grupo como categoria
\newcommand{\sC}{\mathsf{C}}            %% una categoria
\newcommand{\sD}{\mathsf{D}}            %% otra categoria
\newcommand{\sE}{\mathsf{E}}            %% otra categoria mas
\newcommand{\sF}{\mathsf{F}}            %% algebra de Lie F_4
\newcommand{\sG}{\mathsf{G}}            %% algebra de Lie G_2
\newcommand{\sJ}{\mathsf{J}}            %% un poset
\newcommand{\sK}{\mathsf{K}}            %% un poset
\newcommand{\sL}{\mathcal{L}}           %% derivada de Lie
\newcommand{\sN}{\mathsf{N}}            %% categoría con objetos \bN
\newcommand{\sT}{\mathsf{T}}            %% transpuesta

%%% Boldface letters:

\bmdefine{\CC}{C}                       %% C negrilla
\bmdefine{\cc}{c}
\bmdefine{\dd}{d}                       %% d negrilla
\bmdefine{\ee}{e}                       %% vector e
\bmdefine{\eeps}{\varepsilon}           %% basic form \eps
\bmdefine{\FF}{F}                       %% vector F
\bmdefine{\ff}{f}                       %% vector f
\bmdefine{\ii}{i}                       %% cuaternion i
\bmdefine{\jj}{j}                       %% cuaternion j
\bmdefine{\kk}{k}                       %% cuaternion k
\bmdefine{\lla}{\lambda}                %% sucesion \la
\bmdefine{\mmu}{\mu}                    %% sucesion \mu
\bmdefine{\pp}{p}                       %% vector p
\bmdefine{\qq}{q}                       %% vector q
\bmdefine{\rr}{r}                       %% vector r
\bmdefine{\ssg}{\sigma}                 %% vector \sg
%\bmdefine{\sss}{s}
%\bmdefine{\ttt}{t}
\bmdefine{\VV}{V}                       %% V negrilla
\bmdefine{\xx}{x}                       %% sucesion x
\bmdefine{\xxi}{\xi}                    %% vector \xi
\bmdefine{\yy}{y}                       %% sucesion y
\bmdefine{\zz}{z}                       %% sucesion z

% Matrix groups
\DeclareMathOperator{\GL}{GL}   %%% grupo general lineal
\DeclareMathOperator{\Or}{O}    %%% grupo ortogonal
\DeclareMathOperator{\PGL}{PGL} %%% grupo proyectivo lineal
\DeclareMathOperator{\PSL}{PSL} %%% grupo proyectivo lineal especial
\DeclareMathOperator{\PSO}{PSO} %%% grupo proyectivo ortogonal
\DeclareMathOperator{\PSU}{PSU} %%% grupo proyectivo unitario
\DeclareMathOperator{\SL}{SL}   %%% grupo especial lineal
\DeclareMathOperator{\SO}{SO}   %%% grupo especial ortogonal
\DeclareMathOperator{\SU}{SU}   %%% grupo especial unitario

% Numericc
\newcommand{\argmin}{\text{argm\'in}}

%% Brackets
\newcommand{\conj}[1]{\left\lbrace#1\right\rbrace}
\newcommand{\bonj}[1]{\left\lbrack#1\right\rbrack}
\newcommand{\obonj}[1]{\left\rbrack#1\right\lbrack}
\newcommand{\rbonj}[1]{\left\rbrack#1\right\rbrack}
\newcommand{\lbonj}[1]{\left\lbrack#1\right\lbrack}
\newcommand{\snm}[1]{\|#1\|}           %small norma
\newcommand{\nm}[1]{\left\|#1\right\|} %norma pegadita
\newcommand{\pnm}[1]{\biggl|\biggl|#1\biggr|\biggr|}
\newcommand{\ipd}[1]{\left\langle #1\right\rangle}
\let\oldvec=\vec
\renewcommand{\vec}[1]{\mathbf{#1}}
\newcommand\quot[2]{
        \mathchoice
            {% \displaystyle
                \text{\raise1ex\hbox{$#1$}\Big/\lower1ex\hbox{$#2$}}%
            }
            {% \textstyle
                {^{ #1}/_{ #2}}
            }
            {% \scriptstyle
                {^{ #1}/_{ #2}}
            }
            {% \scriptscriptstyle
                {^{ #1}/_{ #2}}
            }
    }
%\newcommand*\quot[2]{{^{\textstyle #1}\big/_{\textstyle #2}}}
\newcommand*\squot[2]{{^{ #1}/_{ #2}}}%%%small quotient

% Probability
\DeclareMathOperator{\Bernoulli}{Bernoulli}
\DeclareMathOperator{\betaD}{beta}
\DeclareMathOperator{\bias}{bias}
\DeclareMathOperator{\binomial}{binomial}
\DeclareMathOperator{\corr}{corr}
\DeclareMathOperator{\cov}{cov}
\DeclareMathOperator{\gammaD}{gamma}
\DeclareMathOperator{\mse}{mse}
\DeclareMathOperator{\multinomial}{multinomial}
\DeclareMathOperator{\Poisson}{Poisson}
\DeclareMathOperator{\Var}{Var}     %%%variance
\DeclareMathOperator{\Cov}{Cov}     %%%Covariance
\renewcommand{\mid}{\;\ifnum\currentgrouptype=16 \middle\fi|\;}

% Algebra
\DeclareMathOperator{\Ad}{Ad}       %% acción adjunta
\DeclareMathOperator{\adj}{adj}
\DeclareMathOperator{\Ann}{Ann}     %% aniquilador o anulador de módulos
\DeclareMathOperator{\Ass}{Ass}     %% ideales asociados
\DeclareMathOperator{\Aut}{Aut}
\DeclareMathOperator{\Char}{char}
\DeclareMathOperator{\codim}{codim}
\DeclareMathOperator{\disc}{disc}
\DeclareMathOperator{\dom}{dom}
\DeclareMathOperator{\End}{End}     %%%space of endomorphisms
\DeclareMathOperator{\Fix}{Fix}
\DeclareMathOperator{\Frac}{Frac}
\DeclareMathOperator{\Gal}{Gal}
\DeclareMathOperator{\gen}{gen}     %%%set generated by...
\DeclareMathOperator{\Hom}{Hom}
\DeclareMathOperator{\image}{image}
\DeclareMathOperator{\Nil}{Nil}
\DeclareMathOperator{\Orb}{Orb}
\DeclareMathOperator{\Quot}{Quot}
\DeclareMathOperator{\Spec}{Spec}
\DeclareMathOperator{\Stab}{Stab}

% Analysis
\DeclareMathOperator*{\esssup}{ess\hspace{0.5mm}sup}
\DeclareMathOperator*{\essinf}{ess\hspace{0.5mm}inf}
%\DeclareMathOperator{\Int}{Int}     %%%interior vacilon funcional

\newcommand{\loc}{\text{loc}}
\newcommand{\LB}{\cL_\cB}           %%%bounded linear operator

% Logic
\newcommand{\cleq}{\preccurlyeq}
\newcommand{\cgeq}{\succcurlyeq}

% Others
\DeclareMathOperator{\ebal}{ev}     %%%evalutation
\newcommand{\bigcupdot}{\charfusion[\mathop]{\bigcup}{\Cdot}} %%JCVDG
%\renewcommand{\bigcupdot}{\charfusion[\mathop]{\bigcup}{\Cdot}}
\newcommand{\cupdot}{\charfusion[\mathbin]{\cup}{\Cdot}}
\newcommand{\exterior}{\mathchoice{{\textstyle\bigwedge}}{{\bigwedge}}{{\textstyle\wedge}}{{\scriptstyle\wedge}}}
\newcommand{\hol}{\mathfrak{hol}}
\newcommand{\Id}{\mathrm{Id}}
\newcommand{\lie}[1]{\mathfrak{#1}}
\newcommand{\qeq}{\mathrel{``{=}"}}
\newcommand{\wsto}{\stackrel{\mathrm{w}^*}{\to}}
\newcommand{\wt}{\mathrm{wt}}

\let\Im\relax
\let\Re\relax

\DeclareMathOperator{\area}{area}
\DeclareMathOperator{\card}{card}
\DeclareMathOperator{\ccl}{ccl}
\DeclareMathOperator{\ch}{ch}
\DeclareMathOperator{\cl}{cl}
\DeclareMathOperator{\coker}{coker}
\DeclareMathOperator{\Conv}{Conv}   %%Convex hull
\DeclareMathOperator{\cosec}{cosec}
\DeclareMathOperator{\cosech}{cosech}
\DeclareMathOperator{\covol}{covol}
\DeclareMathOperator{\diag}{diag}
\DeclareMathOperator{\diam}{diam}
\DeclareMathOperator{\Diff}{Diff}
\DeclareMathOperator{\energy}{energy}
\DeclareMathOperator{\erfc}{erfc}
\DeclareMathOperator{\Ext}{Ext}
\DeclareMathOperator{\fst}{fst}
\DeclareMathOperator{\Fit}{Fit}
\DeclareMathOperator{\gr}{gr}
\DeclareMathOperator{\hcf}{hcf}
\DeclareMathOperator{\id}{id}
\DeclareMathOperator{\Im}{Im}
\DeclareMathOperator{\Ind}{Ind}
\DeclareMathOperator{\Int}{Int}
\DeclareMathOperator{\Isom}{Isom}
\DeclareMathOperator{\lcm}{lcm}
\DeclareMathOperator{\length}{length}
\DeclareMathOperator{\Lie}{Lie}
\DeclareMathOperator{\like}{like}
\DeclareMathOperator{\Lk}{Lk}
\DeclareMathOperator{\Maps}{Maps}
\DeclareMathOperator{\mcd}{mcd}
\DeclareMathOperator{\mcm}{mcm}
\DeclareMathOperator{\Min}{Min}
\DeclareMathOperator{\orb}{orb}
\DeclareMathOperator{\ord}{ord}
\DeclareMathOperator{\otp}{otp}
\DeclareMathOperator{\pr}{pr}       %% proyector
\DeclareMathOperator{\poly}{poly}
\DeclareMathOperator{\rel}{rel}
\DeclareMathOperator{\Rad}{Rad}
\DeclareMathOperator{\Re}{Re}
\DeclareMathOperator*{\res}{res}
\DeclareMathOperator{\Ric}{Ric}
\DeclareMathOperator{\rk}{rk}
\DeclareMathOperator{\Rees}{Rees}
\DeclareMathOperator{\Root}{Root}
\DeclareMathOperator{\spn}{span}
\DeclareMathOperator{\St}{St}
\DeclareMathOperator{\supp}{supp}
\DeclareMathOperator{\Syl}{Syl}
\DeclareMathOperator{\Sym}{Sym}
\DeclareMathOperator{\vol}{vol}

%%% Shorter symbol names:

\newcommand{\bull}{{\scriptstyle\bullet}}  %% vertice en figuras
\newcommand{\del}{\partial}             %% short for  \partial
\newcommand{\downto}{\downarrow}        %% limite a la derecha
\newcommand{\dsp}{\displaystyle}        %% despliegue en texto
\renewcommand{\geq}{\geqslant}          %% mayor o igual (variante)
\newcommand{\hookto}{\hookrightarrow}     %% inclusion arrow
\newcommand{\isom}{\simeq}              %% isomorfismo
\renewcommand{\l}{\ell}                   %% ele cursiva
\renewcommand{\leq}{\leqslant}          %% menor o igual (variante)
\newcommand{\less}{\setminus}           %% set difference
\newcommand{\otto}{\leftrightarrow}     %% bijection
\newcommand{\ox}{\otimes}               %% producto tensorial
\newcommand{\rt}{\triangleleft}         %% un orden parcial
\newcommand{\rteq}{\trianglelefteq}     %% normal subgroup
\newcommand{\up}{{\mathord{\uparrow}}}  %% espinor `up'
\newcommand{\upto}{\uparrow}            %% left hand limit
\newcommand{\w}{\wedge}                 %% producto exterior
\newcommand{\wto}{\rightharpoonup}      %% convergencia debil
\newcommand{\x}{\times}                 %% producto vectorial
\renewcommand{\.}{\Cdot}                %% producto escalar
\renewcommand{\:}{\mathbin{:}}          %% colon in  f: A -> B
\newcommand{\into}{\rightarrowtail}     %% injection arrow
\newcommand{\lr}{\dashv}                %% adjunction
\newcommand{\lt}{\triangleright}        %% a left action
\newcommand{\lteq}{\trianglerighteq}    %% normal supergroup
\newcommand{\nb}{\nabla}                %% homomorfismo de suma
\newcommand{\nisom}{\not\simeq}         %% negacion de isomorfismo
%\newcommand{\oast}{\circledast}         %% variante de * (ya existe en stmaryrd)
\newcommand{\onto}{\twoheadrightarrow}  %% surjection arrow
\newcommand{\opp}{\circ}                %% objeto opuesto
\newcommand{\ottto}{\longleftrightarrow} %% bijection in display
\newcommand{\pullb}{\lrcorner}          %% simbolo de pullback
\newcommand{\pushf}{\ulcorner}          %% simbolo de pushout
\newcommand{\rx}{\rtimes}               %% producto semidirecto
\newcommand{\To}{\Rightarrow}           %% entre funtores
\newcommand{\tofro}{\rightleftarrows}   %% pair of opposed maps
\newcommand{\toto}{\rightrightarrows}   %% pair of parallel maps

\renewcommand{\2}{\flat}                  %% marcador de sucesiones
\newcommand{\3}{\sharp}                 %% marcador de sucesiones
\newcommand{\4}{\natural}               %% marcador de morfismos
% \newcommand{\5}{\diamond}               %% for roots of trees
% \newcommand{\7}{\dagger}                %% adjunto de operador
\newcommand{\8}{\bullet}                %% anonymous degree

%%% Useful abbreviations:

\newcommand{\Coo}{\cC^\infty}         %% funciones suaves
\newcommand{\ctr}{\mathbin{\lrcorner\,}} %% contraction symbol
\newcommand{\nbf}{{\vec\nabla}}     %% short for  \vec\nabla

\newcommand{\as}{\quad\text{cuando}\enspace} %% `cuando' en límites
\newcommand{\bCoo}{{\bC_\infty}}    %% esfera de Riemann
% \newcommand{\bRoo}{{\bR_\infty}}    %% círculo real extendido

%%% Repeated relations:

\newcommand{\cupycup}{\cup\cdots\cup} %% unión repetida
\newcommand{\capycap}{\cap\cdots\cap} %% intersección repetida
\newcommand{\sys}{\subset\cdots\subset}%% subconjunto propio repetido
\newcommand{\subysub}{\subseteq\cdots\subseteq} %%subconjunto repetido
\newcommand{\oxyox}{\otimes\cdots\otimes} %% prod tensorlal repetido
\newcommand{\wyw}{\wedge\cdots\wedge} %% producto exterior repetido
\newcommand{\opyop}{\oplus\cdots\oplus} %% suma directa repetida
\newcommand{\xyx}{\times\cdots\times} %% producto directo repetido

%%% Arrows with riders:

\newcommand{\longto}{\mathop{\longrightarrow}\limits}

%%% Small fractions in displays:

\newcommand{\half}{{\mathchoice{\nhalf}{\thalf}{\shalf}{\shalf}}} %%display text script script^2
\newcommand{\happi}{{\tfrac{\pi}{2}}} %% small fraction  \pi/2
\newcommand{\quarter}{\tfrac{1}{4}} %% small fraction  1/4
\newcommand{\nhalf}{\frac{1}{2}}
\newcommand{\shalf}{{\scriptstyle\frac{1}{2}}} %% tiny fraction 1/2
\newcommand{\thalf}{{\tfrac{1}{2}}} %% small fraction  1/2
\renewcommand{\third}{\tfrac{1}{3}}   %% small fraction  1/3 %Hay que renew porque mathabx toma second y third como x'' y x''' por ejemplo

\newcommand{\ihalf}{{\tfrac{i}{2}}} %% small fraction  i/2

%%%%%%%%%%%%%%%%%%%%%%%%%%%%%
%% 5. Commands with arguments
%%%%%%%%%%%%%%%%%%%%%%%%%%%%%

%%% Accent-like commands, abbreviated:

\newcommand{\ov}{\overline}        %% short for  \overline
\newcommand{\un}{\underline}       %% short for  \underline
\newcommand{\wh}{\widehat}          %% short for  \widehat

%%% Separate words in displays:

\newcommand{\word}[1]{\quad\text{#1}\quad} %% texto intercalado

%%% Webpage locator:

\newcommand{\zelda}[1]{$\langle${\footnotesize\texttt{#1}}$\rangle$}

%%% Proof-part labels:

\newcommand{\Adiff}[2]{\ensuremath{\Ad\,(\mathrm{#1})\Longleftrightarrow
    (\mathrm{#2})}:\enspace}
\newcommand{\Adimp}[2]{\ensuremath{\Ad\,(\mathrm{#1})\Longrightarrow
    (\mathrm{#2})}:\enspace}
\newcommand{\Adit}[1]{\ensuremath{\Ad\,(\mathrm{#1})}:\enspace}

%%% Enclose one argument with delimiters:

\newcommand{\bool}[1]{\llbracket#1\rrbracket} %% condición booleana
\newcommand{\combo}[1]{\operatorname{co}(#1)} %% convex combo
\newcommand{\lin}[1]{\operatorname{lin}\langle#1\rangle} %% `span'
\newcommand{\set}[1]{\{\,#1\,\}}    %% set notation

\newcommand{\floor}[1]{\lfloor#1\rfloor} %% mayor entero <= x
\newcommand{\Set}[1]{\biggl\{\,#1\,\biggr\}} %% set notation (large)
\newcommand{\roof}[1]{\lceil#1\rceil} %% menor entero >= x
\def\genr<#1>{\langle#1\rangle}     %% grupo generado por #1

%%% Asides:

\newcommand{\aside}[1]{$\llbracket$\,#1\,$\rrbracket$} % nota lateral
\newcommand{\hint}[1]{$\llbracket$\,In\-di\-ca\-ci\'on: #1\,$\rrbracket$}

%%% Matrices:

\newcommand{\twobyone}[2]{\begin{pmatrix} %% 2 x 1 matrix
   #1 \\ #2 \end{pmatrix}}
\newcommand{\twobytwo}[4]{\begin{pmatrix} %% 2 x 2 matrix
   #1 & #2 \\ #3 & #4 \end{pmatrix}}
\newcommand{\threebyone}[3]{\begin{pmatrix} %% 2 x 1 matrix
   #1 \\ #2 \\ #3 \end{pmatrix}}
\newcommand{\threebythree}[9]{\begin{pmatrix} %% 3 x 3 matrix
   #1 & #2 & #3 \\ #4 & #5 & #6 \\ #7 & #8 & #9 \end{pmatrix}}
\newcommand{\fourbyone}[4]{\begin{pmatrix} %% 2 x 1 matrix
   #1 \\ #2 \\ #3 \\ #4 \end{pmatrix}}

%%%%%%%%%%%%%%%%%%%%%%%%%%%%
%% 6. Hyphenation exceptions
%%%%%%%%%%%%%%%%%%%%%%%%%%%%

\hyphenation{auto-va-lor auto-va-lo-res auto-vec-tor auto-vec-to-res
car-di-na-li-dad ce-rra-da ce-rra-do ce-rra-das ce-rra-dos cons-tan-te
cons-tan-tes cons-truc-ci cons-truir con-ti-nua con-ti-nua-mente
con-ti-nuas con-ti-nui-dad con-ti-nuo con-ti-nuos co-rres-pon-den-cia
co-rres-pon-de co-rres-pon-den co-rres-pon-dien-te
co-rres-pon-dien-tes co-va-rian-te cual-quier cual-quiera
cu-bri-mien-to desa-rro-lla-do desa-rro-llar des-pu dia-go-nal
dia-go-na-les di-fe-ren-cia-ble di-fe-ren-cia-bles di-fe-ren-cial
di-fe-ren-cia-les di-fe-ren-te di-fe-ren-tes dis-cre-ta dis-cre-tas
dis-cre-to dis-cre-tos di-vi-si-bi-li-dad di-vi-si-ble ele-men-tal
ele-men-ta-les ele-men-to ele-men-tos equi-va-len-cia equi-va-lente
equi-va-lentes equi-va-rian-te equi-va-rian-tes eu-cli-dia-na
eu-cli-dia-nas eu-cli-dia-no eu-cli-dia-nos Fi-gu-ra Gal-ois
gal-oi-sia-na ge-ne-rada ge-ne-rado ge-ne-ra-dor ge-ne-ra-do-res
ge-ne-ral ge-ne-ra-les ge-ne-ra-li-dad ge-ne-ra-li-za ge-ne-ra-li-zan
ge-ne-ran ge-ne-rar geo-me-tr geo-me-try Ha-da-mard ho-meo-mor-fis-mo
ho-meo-mor-fo idea-les in-de-pen-dien-te in-de-pen-dien-tes
in-va-rian-cia in-va-rian-te in-va-rian-tes li-ne-a-les
li-ne-al-men-te ma-ne-ra me-dian-te mo-der-no nin-gu-no nues-tra
nues-tro nu-me-ra-ble ope-ra-ci ope-ra-cio-nes ope-ra-dor
ope-ra-do-res or-to-go-nal par-ti-cu-lar pro-ce-di-mien-to pro-duc-to
pro-duc-tos pro-pie-dad pro-pie-da-des pro-po-si-ci re-fe-ren-cia
re-fle-xi-va re-fle-xi-vas re-fle-xi-vo re-fle-xi-vos re-so-lu-ble
res-pec-ti-va-men-te res-pec-ti-vo res-pec-ti-vos res-pec-to
sa-tis-fa-ce sepa-ra-ble sepa-ra-bles si-guien-te si-guien-tes
subes-pa-cio subes-pa-cios te-dra-edro te-tra-edros tri-vial
tri-via-les uti-lidad va-lo-res va-ria-ble va-ria-bles va-rie-dad
va-rie-da-des ve-cin-da-rio ve-cin-da-rios vec-to-rial vec-to-ria-les
vice-versa}


%%% TikZ arrows and such

\pgfarrowsdeclarecombine{twolatex'}{twolatex'}{latex'}{latex'}{latex'}{latex'}
\tikzset{->/.style = {decoration={markings,
                                  mark=at position 1 with {\arrow[scale=2]{latex'}}},
                      postaction={decorate}}}
\tikzset{<-/.style = {decoration={markings,
                                  mark=at position 0 with {\arrowreversed[scale=2]{latex'}}},
                      postaction={decorate}}}
\tikzset{<->/.style = {decoration={markings,
                                   mark=at position 0 with {\arrowreversed[scale=2]{latex'}},
                                   mark=at position 1 with {\arrow[scale=2]{latex'}}},
                       postaction={decorate}}}
\tikzset{->-/.style = {decoration={markings,
                                   mark=at position #1 with {\arrow[scale=2]{latex'}}},
                       postaction={decorate}}}
\tikzset{-<-/.style = {decoration={markings,
                                   mark=at position #1 with {\arrowreversed[scale=2]{latex'}}},
                       postaction={decorate}}}
\tikzset{->>/.style = {decoration={markings,
                                  mark=at position 1 with {\arrow[scale=2]{latex'}}},
                      postaction={decorate}}}
\tikzset{<<-/.style = {decoration={markings,
                                  mark=at position 0 with {\arrowreversed[scale=2]{twolatex'}}},
                      postaction={decorate}}}
\tikzset{<<->>/.style = {decoration={markings,
                                   mark=at position 0 with {\arrowreversed[scale=2]{twolatex'}},
                                   mark=at position 1 with {\arrow[scale=2]{twolatex'}}},
                       postaction={decorate}}}
\tikzset{->>-/.style = {decoration={markings,
                                   mark=at position #1 with {\arrow[scale=2]{twolatex'}}},
                       postaction={decorate}}}
\tikzset{-<<-/.style = {decoration={markings,
                                   mark=at position #1 with {\arrowreversed[scale=2]{twolatex'}}},
                       postaction={decorate}}}

\tikzset{circ/.style = {fill, circle, inner sep = 0, minimum size = 3}}
\tikzset{scirc/.style = {fill, circle, inner sep = 0, minimum size = 1.5}}
\tikzset{mstate/.style={circle, draw, blue, text=black, minimum width=0.7cm}}

\tikzset{eqpic/.style={baseline={([yshift=-.5ex]current bounding box.center)}}}
\tikzset{commutative diagrams/.cd,cdmap/.style={/tikz/column 1/.append style={anchor=base east},/tikz/column 2/.append style={anchor=base west},row sep=tiny}}

\definecolor{mblue}{rgb}{0.2, 0.3, 0.8}
\definecolor{morange}{rgb}{1, 0.5, 0}
\definecolor{mgreen}{rgb}{0.1, 0.4, 0.2}
\definecolor{mred}{rgb}{0.5, 0, 0}

\def\drawcirculararc(#1,#2)(#3,#4)(#5,#6){%
    \pgfmathsetmacro\cA{(#1*#1+#2*#2-#3*#3-#4*#4)/2}%
    \pgfmathsetmacro\cB{(#1*#1+#2*#2-#5*#5-#6*#6)/2}%
    \pgfmathsetmacro\cy{(\cB*(#1-#3)-\cA*(#1-#5))/%
                        ((#2-#6)*(#1-#3)-(#2-#4)*(#1-#5))}%
    \pgfmathsetmacro\cx{(\cA-\cy*(#2-#4))/(#1-#3)}%
    \pgfmathsetmacro\cr{sqrt((#1-\cx)*(#1-\cx)+(#2-\cy)*(#2-\cy))}%
    \pgfmathsetmacro\cA{atan2(#2-\cy,#1-\cx)}%
    \pgfmathsetmacro\cB{atan2(#6-\cy,#5-\cx)}%
    \pgfmathparse{\cB<\cA}%
    \ifnum\pgfmathresult=1
        \pgfmathsetmacro\cB{\cB+360}%
    \fi
    \draw (#1,#2) arc (\cA:\cB:\cr);%
}
\newcommand\getCoord[3]{\newdimen{#1}\newdimen{#2}\pgfextractx{#1}{\pgfpointanchor{#3}{center}}\pgfextracty{#2}{\pgfpointanchor{#3}{center}}}

\newcommand\qedshift{\vspace{-17pt}}
\newcommand\fakeqed{\pushQED{\qed}\qedhere}

\def\Xint#1{\mathchoice
   {\XXint\displaystyle\textstyle{#1}}%
   {\XXint\textstyle\scriptstyle{#1}}%
   {\XXint\scriptstyle\scriptscriptstyle{#1}}%
   {\XXint\scriptscriptstyle\scriptscriptstyle{#1}}%
   \!\int}
\def\XXint#1#2#3{{\setbox0=\hbox{$#1{#2#3}{\int}$}
     \vcenter{\hbox{$#2#3$}}\kern-.5\wd0}}
\def\ddashint{\Xint=}
\def\dashint{\Xint-}

\newcommand\separator{{\centering\rule{2cm}{0.2pt}\vspace{2pt}\par}}

\newenvironment{own}{\color{gray!70!black}}{}

\newcommand\makecenter[1]{\raisebox{-0.5\height}{#1}}

\mathchardef\mdash="2D

\newenvironment{significant}{\begin{center}\begin{minipage}{0.9\textwidth}\centering\em}{\end{minipage}\end{center}}
\DeclareRobustCommand{\rvdots}{%
  \vbox{
    \baselineskip4\p@\lineskiplimit\z@
    \kern-\p@
    \hbox{.}\hbox{.}\hbox{.}
  }}
\DeclareRobustCommand\tph[3]{{\texorpdfstring{#1}{#2}}}
\def\BState{\State\hskip-\ALG@thistlm}

\makeatother 

\begin{document}
%\clearpage
\maketitle
%\thispagestyle{empty}
{\small
\setlength{\parindent}{0em}
\setlength{\parskip}{1em}

La teoría de curvas elípticas tiene al día de hoy más de 50 años de ser desarrollada continuamente. Fermat estudió ecuaciones que tenían relación  con curvas elípticas y es hasta el día de hoy que se pueden ver las relaciones con estos temas. Este tema se puede ver desde las distintas areas de la matemática: análisis, álgebra, geometría... Hay métodos para el uso de firmas digitales y seguridad en tarjetas de crédito que están relacionados con curvas elípticas.\par
En fín, este tema es de mucho interés en la actualidad. Nosotros nos centraremos en los números racionales siguiendo la línea del libro de Silverman \& Tate \cite{SilvermanTate}. En concreto los temas a tratar son los siguientes:
\begin{enumerate}
  \item Introducción a geometría proyectiva.
  \item Curvas cúbicas y ecuaciones en forma normal de Weierstrass.
  \item Suma de puntos y la ley de grupo en curvas elípticas.
  \item Puntos de orden finito y el teorema Nagell-Lutz.
  \item La estructura del grupo de puntos racionales en una curva elíptica y el teorema Mordell-Weil.
\end{enumerate}


\subsubsection*{Requisitos}
Se asume un conocimiento básico de teoría de números. Se utilizarán conceptos de teoría de grupos y variable compleja a un nivel básico.
}
\newpage
\tableofcontents
%\begin{multicols}{2}

\chapter{Curvas elípticas}

\section{Día 1| 20210105}

\subsection{Introducción a la geometría proyectiva}

\subsubsection{La vista algebraica}

En general hay más de una manera en la que uno puede construir el plano proyectivo y más generalmente el espacio proyectivo en varias dimensiones. A manera de motivación, a la hora de estudiar ciertos problemas en matemática, se llega a observar que es suficiente trabajar con objetos en términos de clases de equivalencia.\par
Por ejemplo un problema que se discute en el Silverman y Tate \cite{SilvermanTate}, al estudiar las soluciones racionales de la ecuación
$$x^N+y^N=1,$$
se puede ver que si $x=\frac ac$ y $y=\frac bd$ son soluciones racionales en su forma más reducida ($\mcd(a,c)=\mcd(b,d)=1$, y $c,d>0$) entonces debe ocurrir que $c=d$. Esto se logra después de un breve análisis de divisibilidad. Concluimos que la solución debe tener la forma $x=\frac ac$ y $y=\frac bc$.\par
Esta solución satisface que $a^N+b^N=c^N$ por lo que la solución $\left(\frac ac,\frac bc\right)$ del problema en términos racionales genera una solución $(a,b,c)$ de la ecuación homogénea $x^N+y^N=z^N$. La clave aquí es que como la ecuación es homogénea, cualquier múltiplo de $(a,b,c)$, $(ta,tb,tc)$ con $t\in\bR$, va a ser una solución al mismo problema. Pero como estas soluciones se obtienen de manera relativamente trivial, podríamos querer considerarlas como equivalentes. Este tipo de razonamiento lleva a la definición algebraica del plano proyectivo.

\begin{Def}
  El \term{plano proyectivo} sobre un cuerpo $K$ es el cociente del conjunto $\set{(a,b,c)\in K^3\less\set{(0,0,0)}}$ por la relación
  $$(a,b,c)\sim (a',b',c')\iff \exists t\in K^\x(a=ta',\ b=tb', c=tc').$$
  Denotamos entonces
  $$\bP^2_K=\quot{\set{x\in K^3\less\set{0}}}{\sim},$$
  y la clase de equivalencia de $(a,b,c)$ la denotamos $[a,b,c]$ y se llamarán sus coordenadas homogéneas.
\end{Def}

\begin{significant}
  ¿Qué pasa si incluimos el cero en nuestra definición del plano proyectivo?
\end{significant}

Bueno, volviendo al ejemplo que presentamos, nos gustaría que nuestras soluciones estén en correspondencia. Claramente $(0,0,0)$ resuelve la ecuación homogénea, pero no la que tiene forma racional. Quizás de manera más interesante, $(1,-1,0)$ resuelve la ecuación homogénea con exponente impar. Pero esta no da una solución de la ecuación racional, entonces podríamos pensar por ejemplo que tenemos $((a_j,b_j,c_j))\subseteq\bR^3$ una sucesión de soluciones reales que converge a $(1,-1,0)$ y $c_j>0$. Esta sucesión si genera soluciones $\left(\frac{a_j}{c_j},\frac{b_j}{c_j}\right)$ de la ecuación racional y cuando $j\to\infty$ entonces este par ordenado tiende a $(\infty,-\infty)$. Podemos entonces pensar que las tripletas con tercera coordenada nula corresponden con soluciones que se encuentran en el infinito. Esta clase de puntos en el infinito es fundamental y lo estudiaremos más adelante.

\begin{Def}
  El \term{espacio proyectivo} en $n$ dimensiones sobre un cuerpo $K$ es el conjunto
  $$\bP^n_K=\quot{\set{x\in K^{n+1}\less\set{0}}}{\sim}$$
  donde la equivalencia es $x\sim x'\iff \exists t\in K^\x(x=tx')$. De igual manera denotamos la clase de $x$ como $[x]$. Las coordenadas de $[x]$ igualmente se llamarán \term{coordenadas homogéneas}.
\end{Def}

Más adelante vamos a definir curvas en el espacio proyectivo. En este momento vamos a definir lo que entenderemos como una recta en el plano proyectivo. En principio verificar que un punto proyectivo $[a,b,c]$ está en una recta proyectiva consiste en ver que cualquier elemento de la clase de equivalencia satisface la ecuación mencionada.

\begin{Def}[Recta en el plano proyectivo]
  Una recta en el plano proyectivo $\bP^2_K$ es el conjunto de puntos $[a,b,c]$ cuyas coordenadas satisfacen una ecuación de la forma
  $$\al X+\bt Y+\ga Z=0,$$
  donde $\al,\bt,\ga\in K$ no son todos nulos.
\end{Def}

En en el plano usual, una recta es el conjunto de puntos que satisface una ecuación $\al x+\bt y+\ga=0$. En el caso proyectivo, la definición de recta que obtuvimos es básicamente lo que obtendríamos de esta ecuación al hacerla homogénea.

\subsubsection{Una visión geométrica}

Sabemos que en $\bR^2$ vale que dos puntos determinan una única recta y similarmente dos rectas se intersecan en un único punto salvo cuando son paralelas. Entonces buscamos extender el concepto de obtener una noción más completa, ¡quisiéramos poder decir que cualesquiera dos rectas se intersecan en un punto!\par 
La idea va a ser asociar a cada recta en el plano una \emph{dirección}. Al hacer esto, estaríamos agregándole un poco más de \emph{información} a una recta. Entonces una recta va a ser el conjunto de puntos \emph{y una dirección}. Para nosotros, dos rectas paralelas tendrán la misma dirección y como decimos ahora que la dirección es parte de la recta, las paralelas \emph{coinciden en la intersección}.\par 
Extendemos el concepto de recta agregando ``la dirección'' como un punto. Si trabajamos sobre $\bR$, habrá que agregar un número infinito de puntos. Porque si agregamos sólo un punto en el infinito como dirección entonces dos pares de rectas paralelas se intersecan en el mismo punto. En particular dos rectas distintas se intersecarían en dos puntos y eso contradice el hecho de que dos rectas distintas se intersecan en un único punto. Esto nos lleva a una idea geométrica para definir el plano proyectivo. 

\begin{Def}[Plano afín sobre $K$]
  Sea $K$ un cuerpo, el \term{plano afín} sobre $K$ es el conjunto $\bA^2_K=\set{(x,y)\in K^2}$. 
\end{Def}

\begin{Def}[Espacio afín n-dimensional]
  El \term{espacio afín} sobre un cuerpo $K$ es el conjunto $\bA^n_K=\set{(x_1,x_2,\dots,x_n)\in K^n}$. 
\end{Def}

Observe ahora la diferencia aquí con el plano proyectivo, se necesitaban tres coordenadas. Aquí en el plano afín necesitamos sólo dos. Análogamente en el espacio proyectivo de $n$ dimensiones, necesitamos $n+1$ coordenadas, mientras que en el $n$-espacio afín usamos $n$ coordenadas.

\begin{Def}[Plano proyectivo sobre K (geometricamente)]
  El \term{plano proyectivo} se define como el conjunto 
  $$\bP_K^2=\bA_K^2\cup\set{\text{direcciones en }\bA_K^2}.$$
  Consideramos que dos rectas en $\bA_K^2$ tienen la misma dirección cuando son paralelas.
\end{Def}

Por tanto una dirección se puede considerar como una clase de equivalencia de rectas. Definimos una equivalencia en el conjunto de las rectas en $\bA_K^2$ al considerar dos rectas como equivalentes cuando son paralelas. Así las direcciones en $\bA^2_K$ son las clases de equivalencia de rectas paralelas.\par 
Entonces los puntos de $\bP_K^2$ correspondientes a las direcciones, los llamamos ``puntos en el infinito''. Al conjunto de puntos en el infinito de hecho lo consideramos una recta en $\bP_K^2$, se le llama \term{recta en el infinito}. 
\begin{Ex}
  La noción de punto en el infinito se puede asociar con la idea de ver un par de lineas de tren. Desde un punto de vista, cuando se ve en la dirección de las lineas en una situación adecuada, pareciera que en el horizonte las lineas se tocan.\par 
  \red{TO DO: Agregar imagen vectorizada de lineas de tren tocándose.}
\end{Ex}

Bajo esta versión geométrica, una recta en $\bP_K^2$ se define de una manera distinta.

\begin{Def} [Recta proyectiva (geometricamente)]
  Una recta en $\bP_K^2$ es la unión de puntos de una recta en $\bA_K^2$ con su correspondiente dirección.
\end{Def}

De acuerdo con esta definición, dos rectas en el plano proyectivo sí se intersecan en un \emph{único punto}. Si dos rectas no son paralelas se intersecan en únicamente un punto del plano afín y sus direcciones son distintas.  Entonces el punto en el infinito que agregamos a ambas rectas no coincide. En el caso que las rectas sean paralelas, no se intersecan en ningún punto del plano afín. Pero pertenecen a la misma clase de equivalencia y por tanto tienen el mismo punto en el infinito asociado. El último caso es el de la recta en el infinito que interseca a todas las rectas pues todas llevan una dirección asociada.\par 
Para reconciliar las definiciones establecemos la equivalencia entre ellas. Redefinimos el conjunto de direcciones que originalmente lo consideramos como un conjunto de clases de equivalencia. Basta considerar sólo aquellas rectas que pasan por el origen, de la cual sólo hay una por cada clase de equivalencia. Vamos a usar estas rectas para describir el conjunto de direcciones a manera de ``representantes canónicos''.\par 
Las rectas que pasan por el origen tienen la forma $Ay=Bx$ donde $A,B$ no son ambos cero. Ciertamente $(A,B)$ y $(A',B')$ definen la misma recta cuando existe $t\in K^\x$ tal que $A=tA'$ y $B=tB'$. Aquellos con mente sagaz habrán reconocido esta idea como una que hicimos recién. El conjunto de direcciones en $\bA_K^2$ se puede describir como el conjunto de puntos $[A,B]\in\bP_K^1$ donde 
$$[A,B]=\set{(tA,tB)\in K^2\less\set{(0,0)},\ t\in K^\x}.$$
Concluimos que $\bP_K^2=\bA_K^2\cup\bP_K^1$ donde $[A,B]$ corresponde con la dirección de la recta $Ay=Bx$. De manera totalmente análoga se puede ver que $\bP_K^n=\bA_K^n\cup\bP_K^{n-1}$.


\section{Día n+1| 20210113}

Hemos concluido la lección anterior observando ejemplos de intersecciones entre curvas. Para lograr que dos curvas de grados $d_1$ y $d_2$ se intersecaran en $d_1\.d_2$ puntos, necesitábamos trabajar tanto en el plano proyectivo como en $\bC$. Continuamos con otro ejemplo:

\begin{Ex}
  Consideramos las curvas
  \begin{align*}
     & C_1\:\ x+y=2,\\
     & C_2\:\ x^2+y^2=2.
  \end{align*}
  \begin{figure}[h]
    \centering
   \begin{tikzpicture}
\begin{axis}[xmax=3.5,ymax=3.5, samples=50]
  \addplot[blue, ultra thick] (x,2-x);
  \draw (axis cs:0,0) circle [red, radius=1.41];
\end{axis}
\end{tikzpicture}
    \caption{aaa}\label{aaaa}
  \end{figure}
  La recta $C_1$ interseca al círculo de forma tangencial en el punto $(1,1)$. De hecho si homogenizamos y buscamos las intersecciones de las curvas proyectivas
  \begin{align*}
     & \tilde C_1\:\ x+y=2z,\\
     & \tilde C_2\:\ x^2+y^2=2z^2,
  \end{align*}
  entonces despejamos $z=\frac{x+y}{2}$. Lo que nos lleva a $x^2+y^2=2\left(\frac{x+y}{2}\right)^2$. Despejando vale que $(X-Y)^2=0$ y no podría ocurrir que $X,Y$ son ambos cero y $X=Y$. Esto nos lleva al punto $[X,X,X]=[1,1,1]\in\tilde C_1\cap\tilde C_2$ y este punto corresponde al punto afín que ya habíamos encontrado.\par
  Nosotros esperábamos dos puntos de intersección, pero no hay manera ni siquiera pasando por los complejos. Sólo obtenemos un punto de intersección y esto ocurre porque la recta interseca de manera tangencial. Este problema es el análogo al de una variable, recuerde que el teorema fundamental del álgebra garantiza la existencia cierto número de raíces para los polinomios. Lo que no hemos mencionado es la \emph{multiplicidad} de la raíz. En este caso vemos que la misma ecuación $(X-Y)^2=0$ nos dice que la raíz tiene multiplicidad dos.\par
  En conclusión sólo encontramos un punto de intersección, incluso después de haber buscado posibles puntos en el infinito. Esto es porque $[1,1,1]$ se debe contar con multiplicidad dos y corresponde con el hecho de que ambas curvas se intersecan de manera tangencial en este punto. De hecho, se ve reflejado en que al resolver el sistema de ecuaciones obtuvimos la ecuación $(X-Y)^2=0$.
\end{Ex}

Este no es el único caso en que esto puede ocurrir y el siguiente ejemplo lo ilustra.

\begin{Ex}
  Considere ahora la curvas
  \begin{align*}
     & C_1\:\ y=x,\\
     & C_2\:\ y^2=x^3.
  \end{align*}
    \red{aregar figura} Estas curvas se intersecan en dos puntos del plano afín. Al homogenizar obtenemos
    \begin{align*}
     & \tilde C_1\:\ X-Y=0,\\
     & \tilde C_2\:\ X^3-Y^2Z=0,
  \end{align*}
  y resolviendo llegamos a $X^3-X^2Z=0$ y de aquí que $X=0$ ó $X=Y=Z$. Si $X=0$, entonces $Y=0$ y así $Z$ queda libre lo que nos lleva a los puntos $[1,1,1]$ y $[0,0,1]$. A diferencia del caso anterior, no hay multiplicidades mayores a uno. Aquí lo que ocurre es que uno de los puntos de intersección es un punto que no es suave. Las rectas cuyas direcciones aproximan la identidad tienen dos puntos de intersección con la curva cuspidal, entonces en el límite, el punto singular $[0,0,1]$ tiene multiplicidad dos. \red{agregar figura}
\end{Ex}

\begin{Ex}
  Esta vez consideramos las curvas
  \begin{align*}
     & C_1\:\ x+y+1=0,\\
     & C_2\:\ 2x^2+xy-y^2+4x+y+2=0.
  \end{align*}
  Aquel que esté atento podrá notar que
  $$2x^2+xy-y^2+4x+y+2=(x+y+1)(2x-y+2)$$
  y así $C_1\subseteq C_2$ lo que nos dice que la cantidad de puntos de intersección es infinita. Pero entonces las situaciones así no deben entrar en la consideración de los puntos de intersección de manera tan vaga. En el caso de $\bA^2_\bR$ hay infinitos puntos por lo que debemos especificar lo que buscamos.
\end{Ex}

\begin{Def}
  Sea $C\:\ f(x,y)=0$ una curva con $f\in K[x,y]$. Si factorizamos $f$ como un producto de polinomios irreducibles $f=\prod_{j=1}^{n}p_j$, entonces los \term{componentes} de la curva $C$ son las curvas $C_j\:\ p_j(x,y)=0$. Diremos que $C$ es irreducible cuando tenga un único componente. Es decir, sólo si el polinomio $f$ es irreducible.
\end{Def}

\begin{Ex}
  De las curvas estudiadas en el ejemplo anterior, la curva $C_1$ es irreducible al ser un polinomio lineal y la segunda curva se puede factorizar en dos componentes. Ambas rectas son las componentes irreducibles de esta curva.
\end{Ex}

\begin{Def}
  Diremos que dos curvas afines $C_1, C_2$ no tienen componentes en común si sus componentes irreducibles son distintos.
\end{Def}

Un resultado básico en teoría de curvas que no vamos a demostrar es el siguiente:

\begin{Prop}
  Si $C_1, C_2$ son dos curvas afines sin componentes en común, entonces $C_1\cap C_2$ es un conjunto finito.
\end{Prop}

\begin{Rmk}
  De manera análoga al caso afín, se definen componentes de curvas proyectivas y la noción de dos curvas proyectivas sin componentes comunes.
\end{Rmk}

El teorema de Bezout es más general que este resultado. Con lo que hemos visto hasta ahora, lo podemos enunciar.\red{bajé a agarrar agua}\par
Por ahora mencionamos las siguientes propiedades:
\begin{enumerate}
  \item Si $P\not\in C_1\cap C_2$, entonces $I(C_1\cap C_2,P)=0$.
  \item Si $P\in C_1\cap C_2$ y $P$ es un punto no singular de $C_1$ y $C_2$, y si adicionalmente $C_1$ y $C_2$ tienen direcciones tangenciales diferentes en $P$, entonces $I(C_1\cap C_2,P)=1$. En este caso, se dice que $C_1$ y $C_2$ se intersecan en $P$ de manera transversal.
  \item Si $P\in C_1\cap C_2$ y $C_1$ y $C_2$ no se intersecan transversalmente, entonces $I(C_1\cap C_2,P)\geq 2$.
\end{enumerate}

\begin{Th}[Bezout]
  Sean $C_1,C_2\subseteq\bP_\bC^2$ sin componentes en común. Entonces vale que
  $$\sum_{P\in C_1\cap C_2}I(C_1\cap C_2,P)=\deg(C_1)\deg(C_2).$$
  En particular si $C_1,C_2$ son suaves y únicamente tienen intersecciones transversales, entonces
  $$|C_1\cap C_2|=\deg(C_1)\deg(C_2)$$
  y en todo momento se tiene la desigualdad $|C_1\cap C_2|\leq\deg(C_1)\deg(C_2)$.
\end{Th}

\section{Día n+2| 20210114}

\subsection{Multiplicidad de la intersección de dos curvas}
Vamos a estudiar algunas propiedades y ejemplos de cálculo de la multiplicidad o índice de intersecciones $I(C_1\cap C_2,P)$. Vamos a comenzar con un teorema que establece la existencia de la multiplicidad de la intersección y nos permite hacer cálculos.\par

Rápidamente para poder simplificar la notación introducimos los conceptos de variedad.

\begin{Def}
  La \term{variedad afín} de $f$ es el conjunto de ceros de $f$ dentro del espacio afín $\bA^2$. Denotamos
  $$V(f)=\set{x\in\bA^2\:\ f(x)=0}$$
  y análogamente la \term{variedad proyectiva} de $F$ es su conjunto de ceros dentro del espacio proyectivo. Este conjunto es
  $$V(F)=\set{X\in\bP^2\:\ F(X)=0}.$$
\end{Def}
\begin{Th}
  Considere $V(f),V(g)$ dos curvas afines en $\bA_\bC^2$ y $P\in\bA_\bC^2$ dado. Entonces existe un número $I(V(f)\cap V(g),P)$ definido de manera única tal que las siguientes propiedades se satisfacen:
  \begin{enumerate}
    \item $I(V(f)\cap V(g),P)\in\bZ_{\geq 0}$, a menos que $P$ esté en un componente común de $V(f),V(g)$ y en ese caso $I(V(f)\cap V(g),P)=\infty$.
    \item $I(V(f)\cap V(g),P)=0$ si y sólo si $P\not\in V(f)\cap V(g)$.
    \item Dos rectas distintas se intersecan con multiplicidad uno en su punto de intersección.
    \item $I(V(f)\cap V(g),P)=I(V(g)\cap V(f),P)$.
    \item Si $f=\prod p_i^{\al_i}$ y $g=\prod q_i^{\bt_i}$, entonces
    $$I(V(f)\cap V(g),P)=\sum_{i,j}\al_i\bt_jI(V(p_i)\cap V(q_j),P).$$
    \item $I(V(f)\cap V(g),P)=I(V(f)\cap V(g+hf),P)$ para $h\in\bC[x,y]$.
 \end{enumerate}
\end{Th}

\begin{Def}
  El número $I(V(f)\cap V(g),P)$ se llama \term{multiplicidad de la intersección} de $V(f)$ y $V(g)$ en $P$.
\end{Def}

\begin{Ex}
  \begin{enumerate}
    \item $x^2$ con $y$
    \item círculo con recta
    \item cuspidal con identidad
  \end{enumerate}
\end{Ex}

\begin{Def}
  Sea $f$ un polinomio con coeficientes en $\bC$ y $P\in V(f)$. La \term{multiplicidad} de $f$ en $P$
\end{Def}

\chapter{Análisis de Fourier}

\section{Día k| 20210203}
\subsection{La tranformada de Fourier}

Trabajaremos en $\bR^d$ con $m$ la medida de Lebesgue, las funciones a considerar son $f\:\bR^d\to\bC$ y suponemos que son Borel medibles. De aquí tenemos el espacio
$$L^p(\bR^d)=\Set{f\:\bR^d\to\bC\ \text{medible},\ \int\limits_{\bR^d}|f(x)|^p\dd x<\infty}$$
para $p\in[1,\infty[$. En el caso cuando $p=\infty$ tenemos que $\nm{f}_\infty=\esssup(f)$ como el $\inf_{C>0}\set{m(|f|\geq C)=0}$ y definimos $L^\infty$ de manera análoga a $L^p$ como $\set{\nm{f}_\infty>\infty}$. Trabajamos también con el espacio $C_0$... 
\red{estaba cambiando el word wrap}\par
Así, $f$ es integrable cuando $f_r$ y $f_i$ lo sean.\par 
Si $z\in\bC,\ z=x+iy$ definimos el módulo complejo como $|z|=\sqrt{x^2+y^2}$. El módulo complejo cumple algunas propiedades tales como la desigualdad triangular para integrales $\left|\int f\dd x\right|\leq\int|f|\dd x$. Recordemos la identidad de Euler, si $x\in\bR$ entonces $e^{ix}=\cos(x)+i\sin(x)$ y este número tiene módulo 1.

\begin{Def}
  Sea $f\in L^1(\bR^d)$, definimos su \term{transformada de Fourier} como
  $$\cF(f)(\xi)=\hat f(\xi)=\int\limits_{\bR^d}f(x)e^{-2\pi i\braket{x}{\xi}}\dd x,\ \xi\in\bR^d.$$
  Note que $|\hat{f}(\xi)|\leq \nm{f}_1$.
\end{Def}

\begin{Th}
  El mapeo $f\mapsto\hat f$ es una aplicación continua entre $L^1$ y $L^\infty$.  Vale que $\nm{\hat f}_\infty\leq\nm{f}_1$. Además si $f\in L^1$ entonces $\hat f$ es una función uniformemente continua.
\end{Th}

\begin{ptcbp}
  Note que inmediatamente de la definición tenemos $|\hat{f}(\xi)|\leq\nm{f}_1$ para $\xi\in\bR^d$ lo que implica $\nm{\hat{f}}_\infty\leq\nm{f}_1$.\par
  Por otro lado considere la cantidad
  \begin{align*}
    |\hat{f}(\xi+h)-\hat{f}(\xi)|=\left|\int f(x)e^{-2\pi i\braket{x}{\xi+h}}\dd x-\int f(x)e^{-2\pi i\braket{x}{\xi}}\dd x\right|\\
  \leq \int|f(x)||e^{-2\pi i\braket{x}{\xi}}||e^{-2\pi i\braket{x}{\xi}}-1|\dd x\\
  =\int|f(x)||e^{-2\pi i\braket{x}{\xi}}-1|\dd x.
  \end{align*}
  La cantidad $|e^{-2\pi i\braket{x}{\xi}}-1|$ tiende a cero conforme $h\to 0$ por lo que aplicando el teorema de convergencia dominada tenemos que $|\hat{f}(\xi+h)-\hat{f}(\xi)|\to 0$ cuando $h\to 0$ independiente de $\xi$. 
\end{ptcbp}

\begin{Lem}[Riemann-Lebesgue]
  Si $f\in L^1$, entonces $\hat{f}(\xi)\to 0$ cuando $\nm{\xi}\to\infty$. Es decir, $\hat{f}\in C_0$.
\end{Lem}

\begin{ptcbp}
  Sea $R=\bigtimes_{i\in[d]}[a_i,b_i]\subseteq\bR^d$ un rectángulo en $\bR^d$. Si $f=\ind_R$ entonces tenemos que $f=\prod_{i\in[d]}\ind_{[a_i,b_i]}$.\red{aaaa me perdí}\par
  Vale entonces 
  $$\hat{\ind}_{[a,b]}(\xi)=\int\limits_a^be^{-2\pi ix\xi}\dd x=\int\limits_a^b(\cos(-2\pi x\xi)+i\sin(-2\pi x\xi))\dd x.$$
  Esta cantidad resulta ser $\frac{-1}{2\pi i\xi}(e^{-2\pi ib\xi}-e^{-2\pi ia\xi})$ que tiende a cero cuando $|\xi|\to 0$. Así $\hat{\ind}_R(\xi)\to 0$ cuando $\nm{\xi}\to\infty$.\par
  Para $f\in L^1$ en general, aproximamos con funciones simples cuyas indicadoras son sobre rectángulos.
\end{ptcbp}

Recuerde que la convolución de funciones en $L^1$ es
$$f\ast g(x)=\int f(x-y)g(y)\dd y.$$
La convolución es cerrada en $L^1$, es asociativa y conmutativa.

\begin{Th}
  Tome $f\in L^p, g\in L^1$ para $p\in [1,\infty]$. Entonces $f\ast g\in L^p$ y $\nm{f\ast g}_p\leq\nm{f}_p\nm{g}_1$.
\end{Th}

De este resultado extraemos que la convolución hereda las propiedades más bonitas de sus operandos. 

\begin{ptcbp}
  Por Minkowski tenemos 
  $$\left(\right)$$
  \red{finish}
\end{ptcbp}

\begin{Prop}
  Para $f,g\in L^1$ vale que:
  \begin{enumerate}
    \item $\cF(f\ast g)=\cF(f)\cF(g)$.
    \item $\cF(\tau_h f)(\xi)=e^{2\pi i\braket{h}{\xi}}\cF(f)(\xi)$. Es decir, traslación se convierte en modulación.
    \item Si $A\in O(d)$, el grupo ortogonal en $d$ dimensiones, entonces
    $$\cF(f(A\.))(\xi)=\cF(f(A\xi)).$$
    \item Si $f_\la(x)=\frac{1}{\la^d}\left(\frac{x}{\la}\right)$ entonces $\cF(f_\la)(\xi)=\cF(f)(\la\xi)$.
    \item $\cF\left(\pdv{f}{x_j}\right)(\xi)=2\pi i\xi_j\cF(f)(\xi)$, cuando $f_j\in L^1$. Y para el otro lado, $\cF(-2\pi ix_j f)(\xi)=\cF\left(\pdv{f}{\xi_j}\right)(\xi)$ cuando $x_jf\in L^1$.
  \end{enumerate}
\end{Prop}

Note que a partir del último punto, podemos iterar con derivadas de orden superior cuando todo esté bien definido. Vale por ejemplo que
$$\cF\left(\pdv[2]{f}{x_j}{x_k}\right)(\xi)=(2\pi i\xi_j)(2\pi i\xi_k)\cF(f)(\xi).$$
Naturalmente nos preguntamos, 
\begin{significant}
  ¿Se puede recuperar $f$ por medio de $\cF(f)$?
\end{significant}
Cuando se trabaja con series de Fourier, se puede recuperar $f$ por medio de sus coeficientes de Fourier. Lo esperado es que $f(x)=\int\hat{f}(\xi)e^{2\pi i\braket{x}{\xi}}\dd\xi$. Pero $\hat{f}$ no necesariamente es integrable.

\subsection{La clase de Schwartz}

Indistintamente hablaremos de la derivada parcial de $f$ respecto a $x_j$ como $\del_j f$ o $D_j f$. La $m$-ésima derivada será en su lugar $\del_j^m f$ o $D_j^m f$. Introducimos brevemente la notación multi-índice, si $\al\in\bN^d$ entonces
$$D^\al f=D_1^{\al_1}\.\dots\.D_d^{\al_d}f.$$
Por ejemplo $D^{(3,1,4)}f=\frac{\del^8f}{\del x_1^3\del x_2\del x_3^4}$. También tenemos $\al!=\prod\al_j!$ y $x^\al=\prod x_j^{\al_j}$.

\begin{Def}
  Una función $f$ está en la \term{clase de Schwartz} $\cS(\bR^d)$ si es infinitamente diferenciable y todas sus derivadas decrecen rápidamente a infinito. Es decir para $\al,\bt$ multi-índices vale
  $$\sup_{\bR^d}|x^\al D^\bt f(x)|=\rho_{\al,\bt}(f)<\infty.$$
\end{Def}

Inmediatamente de la definición $\Coo_c\subseteq\cS$, también $\rho_{\al,\bt}$ es una seminorma lo que nos puede llevar a una topología. El conjunto $\set{\rho_{\al,\bt}}$ es una familia contable. Entonces $(f_n)\to 0$ en $\cS$ cuando $\rho_{\al,\bt}(f_n)\to 0$ para $\al,\bt$ son cualquier multi-índice. Podemos ver que $\cS$ es un espacio de Fréchet, la métrica que se puede definir es
$$d(f,g)=\sum_{j\in\bN}\frac{\rho_j(f-g)}{2^j(1+\rho_j(f-g))},$$
donde $(\rho_j)$ es una enumeración de las seminormas.

\begin{Th}
  Ocurre que $\cS$ es denso en $L^p$ para $p\in[1,\infty[$.
\end{Th}

Note que si valiese que $\cS\subseteq L^p$ entonces, como $\Coo_c\subseteq\cS$ y $\Coo_c$ es denso ya entonces estaríamos listos. Basta probar que el espacio de Schwartz está en $L^p$.

\begin{ptcbp}
  Tome $f\in\cS$, entonces 
  \begin{align*}
    \int|f|^p\dd x&=\int(1+|x|^{d+1})\frac{|f|}{1+|x|^{d+1}}\\
    &\leq \int\frac{c(|f|+\sum_{j\in[d]}|x_j|^{2d+2}|f|)^p}{(1+|x|^{d+1})^p}\dd x\\
    &\leq \int\frac{c(\rho_{0,0}f+\sum_{j\in[d]}\rho_{(2d+2)e_j,0}f)^p}{(1+|x|^{d+1})^p}\dd x\\
    &\leq c_2\int\frac{\dd x}{(1+|x|^{d+1})^p}<\infty
  \end{align*} 
  \red{no terminé}
\end{ptcbp}

En particular, el espacio de Schwartz está contenido en $L^1$ y por tanto podemos definir la transformada de Fourier en $\cS$.

\begin{Th}
  El mapeo $f\mapsto\hat f$ es continuo sobre $\cS$ y cumple:
  \begin{enumerate}
    \item $\int f\hat g=\int\hat fg$.
    \item Vale la fórmula de inversión
     $$f(x)=\int\hat f(\xi)e^{2\pi i\braket{x}{\xi}}\dd\xi.$$
  \end{enumerate}
\end{Th}

Precisamos un par de lemas antes de probar este resultado. Note que la fórmula de inversión tiene \emph{casi} las misma forma que la fórmula que la transformada ordinaria. En cierto sentido $f(x)=\hat{\hat f}(-x)$.

\begin{Lem}
Si $f(x)=e^{-\pi\nm{x}^2}$, entonces $\hat{f}=f$. 
\end{Lem}

\begin{ptcbp}
  \red{me salté la prueba, DO}
\end{ptcbp}

Ahora podemos probar el teorema de la fórmula de inversión.

\begin{ptcbp}
  Tenemos que para $f\in\cS$ y $\al,\bt$ multi-índices vale que 
  \begin{align*}
    \xi^\al D^{\bt}\hat{f}(\xi)&=(\prod\xi_j^{\al_j})(D_{\xi_1}^{\bt_1}\dots D_{\xi_d}^{\bt_d})\hat{f}(\xi)\\
    &=C(\cF((D_{\xi_1}^{\bt_1}\dots D_{\xi_d}^{\bt_d})(\prod_{j\in[d]}x_j^{\bt_j})))\dots
  \end{align*}
  $D^\al(x^\bt f)$ es una suma de monomios por derivadas de $f$.
\end{ptcbp}

\section{Día k+1| 20210204}

\subsection{La transformada de Fourier en $L^2$}
\red{anotación sobre el conjugado y extensión a L2.}\par

En $L^2$ por tanto existe una extensión de la transformada de Fourier. Los límites en el enunciado del teorema son consecuencia de la continuidad de la transformada pues lo que tenemos es que $f\ind(B(0,R))\to f$ y $\hat f\ind(B(0,R))\to \hat f$ donde ambas convergen en $L^2$. Esto se conoce como extender el operador por densidad y más adelante veremos cómo extender esto a $L^p$ para $p>2$.

Vimos que en $L^1$ no necesariamente hay inversa de la transformada porque esta no necesariamente es integrable. Tomemos $f\in L^1(\bR^d)$, esperamos que $f(x)=\int\limits_{\bR^d}\hat{f}(\xi)e^{2\pi i\braket{x}{\xi}}\dd\xi$ en algún sentido de convergencia. Sin embargo no estamos asumiendo que $\hat{f}$ es integrable, puede que dicha integral ni siquiera tenga sentido. Aún siendo $f$ la indicadora de un intervalo, la integral en cuestión no existe.

\begin{Ex}
  Si $f=\ind([-a,a])$, entonces $\hat{f}(\xi)=\frac{\sin(\pi a\xi)}{\pi a\xi}$ y esta función no es integrable en el sentido de Lebesgue.
\end{Ex}

Dado esto, necesitamos aplicar métodos de sumabilidad. 

%\subsection{Sumabilidad de Abel}
\begin{Def}
  Sea $\eps>0$, definimos la \term{media de Abel} de $f$ como 
  $$A_\eps f=\int\limits_{\bR^d}$$
\end{Def}

\begin{Def}
  Si definimos $G_\eps f=\int f(x)e^{-\eps\nm{x}^2}\dd x$ para $\eps>0$, entonces decimos que $\int f$ es \term{Gauss sumable} a $\l$ si $\lim_{\eps\to 0}G_\eps f=\l$.
\end{Def}

Ambos promedios se pueden escribir de la forma
$$M_{\eps,\Phi}f=M_\eps f=\int\limits_{\bR^d}\Phi(\eps x)f(x)\dd x$$
con $\Phi\in\cC_0$ y $\Phi(0)=1$.

Entonces la idea es modificar un poco las cosas para obtener cierta convergencia. Para lo que queremos hacer, necesitamos las transformadas de Fourier de $e^{-\eps\nm{x}^2}$ y $e^{-\eps\nm{x}}$. Sabemos que $\cF(e^{-\pi\nm{x}^2})(\xi)=e^{-\pi\nm{\xi}^2}$ por lo que para tener la de $\eps$ basta con hacer una dilatación. Si llamamos $g(x)=e^{-\pi\nm{x}^2}$, entonces $e^{-\eps\nm{x}^2}=g\left(\sqrt{\frac{\eps}{\pi}}x\right)$ y como $\cF(\la^{-d}h(\la^{-1}x))=\hat{h}(\la\xi)$, entonces tendremos que 
$$\cF(e^{-\eps\nm{x}^2})(\xi)=\red{calc}.$$

\begin{Th}
  En general, si $a>0$, tenemos que 
  $$\int\limits_{\bR^d}e^{-\pi a\nm{x}^2}e^{-2\pi i\braket{x}{\xi}}\dd x=a^{-\frac{d}{2}}e^{-\frac{\pi}{a}\nm{\xi}^2}.$$
  También vale que 
  $$\int\limits_{\bR^d}e^{-2\pi a\nm{x}}e^{-2\pi i\braket{x}{\xi}}\dd x=c(d)\frac{a}{(a^2+\nm{\xi}^2)^{\frac{d+1}{2}}}$$
  con $c(d)=\frac{\Ga\left(\frac{d+1}{2}\right)}{\pi^{\frac{d+1}{2}}}$.
\end{Th}

\begin{ptcbp}
  \red{Ejercicio}
\end{ptcbp}

Para simplificarnos los cálculos, llamemos $W=W(\xi,a)=\cF(e^{-4\pi^2a\nm{\.}^2})$ y $P=P(\xi,a)=\cF(e^{-2\pi a\nm{\.}})$. A $W$ lo conocemos como el \term{núcleo de Gauss-Weierstrass} y $P$ como el \term{núcleo de Poisson}.\par 
Así, queremos probar que las medias de Abel y Gauss de $\int\hat{f}\exp(2\pi i\braket{x}{\xi})\dd\xi$ convergen a $f$ en norma y casi por doquier. Esto nos diría que 
$$f(x)=\int\limits_{\bR^d}\hat{f}(\xi)e^{2\pi i\braket{x}{\xi}}w(\xi)\dd\xi$$
lo que nos permite recuperar $f$ por medio de su transformada.\par 
Tomemos $\Phi\in\cC_0\cap L^1$, con $\Phi(0)=1$. Llame $\vf=\hat\Phi$ y $\vf_\la=\la^{-d}\vf(\la^{-1}x)$ para $\la$ positivo. Con esta notación vale que 
\begin{gather*}
  \Phi(x)=\exp(-4\pi^2\nm{x}^2)\To\vf_\eps=W(algo),\\
  \Phi(x)=\exp(-2\pi\nm{x})\To\vf_\eps=P(\xi,\eps).
\end{gather*}
 
\begin{Th}
  Si $f,\Phi\in L^1(\bR^d)$ y $\vf=\hat\Phi$, entonces 
  $$\int\hat{f}(\xi) e^{2\pi i\braket{x}{\xi}}\Phi(\eps\xi)\dd\xi=\int f(x)\vf_\eps(x-\xi)\dd x,\eps>0.$$
  En particular, $\int\hat{f}(\xi)e^{2\pi i\braket{x}{\xi}}e^{-2\pi \eps\nm{\xi}}\dd\xi=\int f(x)P(x-\xi,\eps)\dd x$.
\end{Th}

La prueba del teorema se basa en la fórmula $\int f\hat g=\int\hat fg$ a $f$ y a $\Phi(\eps x)e^{2\pi i\braket{x}{\xi}}$. Resta por notar que los núcleos con los que estamos trabajando, ambos integran a uno. La prueba de estos hechos es un \red{ejercicio}.

\begin{Th}
  Si $\vf\in L^1$ con $\int\vf=1$ y para $\eps>0$ definimos $\vf_\eps(x)=\frac{1}{\eps^d}\vf\left(\frac{x}{\eps}\right)$, entonces si $f\in L^p\cup\cC_0$ para $1\leq p<\infty$, vale que 
  $$\nm{f\ast\vf_\eps-f}_p\to 0,\quad \eps\to 0.$$
  En particular $\int f(\xi)P(x-\xi,\eps)\dd\xi$ y $\int f(\xi)W(x-\xi,\eps)\dd\xi$ convergen a $f$ en $L^p$ cuando $\eps\to 0$.
\end{Th}


%%%%%%%%%%%% Contents end %%%%%%%%%%%%%%%%
\ifx\nextra\undefined
\printindex
\else\fi
\nocite{*}
\bibliographystyle{plain}
\bibliography{bibiTNum}
\end{document} 